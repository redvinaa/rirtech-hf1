\section{BDC motor leírása}

%{{{
\subsection{A BDC motor hatásvázlata}

Legyen
\begin{equation}
	W_\text{el} = \frac{1}{R_\text{a} + L_\text{a}s}
\end{equation}
az elektromos kör átviteli függvénye, és
\begin{equation}
	W_\text{me} = \frac{1}{b + J_\text{a}s} \overset{b=0}{=} \frac{1}{J_\text{a}s}
\end{equation}
a mechanikai kör átviteli függvénye.

Ekkor a rendszer hatásvázlatát \aref{fig:hatasvazlat}. ábra mutatja.

\begin{figure}[h]
	\centering
	\incfig[.8\textwidth]{hatasvazlat}
	\caption{Hatásvázlat}
	\label{fig:hatasvazlat}
\end{figure}
%}}}

%{{ W_u0<>w1
\subsection{A feszültség -- szögsebesség átviteli függvény felírása}

Vegyük \aref{fig:hatasvazlat}. ábrát, és legyen $\tau_\text{t}=0$.
A visszacsatolt kör átviteli függvénye a keresett árviteli függvény:
\begin{equation}
	W_{u_0,\omega_1} =
	\frac{W_\text{el}k_\text{m}W_\text{me}}{1+W_\text{el}k_\text{m}^2W_\text{me}} =
	\frac{k_\text{m}}{(R_\text{a}+L_\text{a})J_\text{a}s+k_\text{m}^2}.
\end{equation}
%}}
