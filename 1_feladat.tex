\section{BDC motor leírása}

%{{{ A BDC motor hatásvázlata
\subsection{A BDC motor hatásvázlata}

Legyen
\begin{equation}
	\fn{W}_\text{el} = \frac{1}{\Ra  + \La s}
\end{equation}
az elektromos kör átviteli függvénye, és
\begin{equation}
	\fn{W}_\text{me} = \frac{1}{b + \Ja s} \overset{b=0}{=} \frac{1}{\Ja s}
\end{equation}
a mechanikai kör átviteli függvénye.

Ekkor a rendszer hatásvázlatát \aref{fig:hatasvazlat}. ábra mutatja.
$\fn{U}_0$ a motorra kapcsolt feszültség, amelyből levonjuk a szögsebességgel arányos feszültséget.
Az így kapott feszültség, $\fn{U}_\text{be}$ esik az ellenálláson. $\fn{W}_\text{el}$ az armatúra
ellenállás és tekercs admittanciája, amelyre rákapcsolva $\fn{U}_\text{be}$-t megkapjuk az
armatúra áramot. A motor nyomatékát a $\km$ szorzó adja, a $\tau_\text{v}=\km \fn{I}_\text{a}$
egyenlet alapján, amihez előjelesen hozzáadjuk a külső terhelést. Az így kapott összes
nyomatékot a $\fn{W}_\text{me}$-vel szorozva kapjuk meg a motor szögsebességét,
$\Omega_\text{ki}$-t, ami tartalmazza a forgórész tehetetlenségi nyomatékát, valamint
a csapágysúrlódást is, amit most elhanyagolunk. $\Omega_\text{ki}\ke$ adja a feszültséget, 
amit ki kell vonnunk a kapocsfeszültségből.

\begin{figure}[h]
	\centering
	\incfig[.8\textwidth]{hatasvazlat}
	\caption{Hatásvázlat}
	\label{fig:hatasvazlat}
\end{figure}
%}}}

%{{{ A feszültség -- szögsebesség átviteli függvény felírása
\subsection{A feszültség -- szögsebesség átviteli függvény felírása}
\label{subsect:Wu-w}

Vegyük \aref{fig:hatasvazlat}. ábrát, és legyen $\tau_\text{t}=0$.

Az előrecsatoló és visszacsatoló át átviteli függvényei
\begin{equation}
	\fn{W}_\text{x} = \fn{W}_\text{el}\km\fn{W}_\text{me},
\end{equation} és 
\begin{equation}
	\fn{W}_\text{f} = \ke.
\end{equation}

A zárthurkú átviteli függvény
\begin{equation}
	\fn{W}_{\fn{U}_0\rightarrow\Omega_\text{ki}} =
	\frac{\fn{W}_\text{x}}{1 + \fn{W}_\text{x}\fn{W}_f} =
	\frac{\km }{\La\Ja s^2 + \Ja\Ra s + \ke\km}
\end{equation}

A $\fn{W}_{\fn{U}_0\rightarrow\Omega_\text{ki}}$ pólusai:
\begin{equation}
	p_1 = \frac{\sqrt{\Ja(\Ja\Ra^2 - 4\La\ke\km)} - \Ja\Ra}{2\Ja\La}
\end{equation}
\begin{equation}
	p_2 = -\frac{\sqrt{\Ja(\Ja\Ra^2 - 4\La\ke\km)} + \Ja\Ra}{2\Ja\La}
\end{equation}

Az időállandók a pólusok reciprokának negáltjai:
\begin{equation}\label{eq:T1}
	T_1 = \frac{2\Ja\La}{\sqrt{\Ja(\Ja\Ra^2 - 4\La\ke\km)} + \Ja\Ra} = 0,0145\text{ s}
\end{equation}
\begin{equation}
	T_2 = -\frac{2\Ja\La}{\sqrt{\Ja(\Ja\Ra^2 - 4\La\ke\km)} - \Ja\Ra} = 1,3825\cdot 10^{-4}\text{ s}
\end{equation}

A rendszer nullfrekvenciás erősítése
\begin{equation}
	A_{\fn{U}_0\rightarrow\Omega_\text{ki}} = \frac{1}{k_\text{e}} = 17,1762
\end{equation}

%}}}

%{{{ A feszültség -- áram átviteli függvény felírása
\subsection{A feszültség -- áram átviteli függvény felírása}
\label{subsect:Wu-i}

Az előrecsatoló és visszacsatoló át átviteli függvényei
\begin{equation}
	\fn{W}_\text{x} = \fn{W}_\text{el},
\end{equation} és 
\begin{equation}
	\fn{W}_\text{f} = \km\ke\fn{W}_\text{me}.
\end{equation}

A zárt hurkú átviteli függvény
\begin{equation}
	\fn{W}_{\fn{U}_0\rightarrow \fn{I}_\text{a}} = 
	\frac{\Ja\, s}{\Ja\, \La\, s^2 + \Ja\, \Ra\, s + \ke\, \km}
\end{equation}

A $\fn{W}_{\fn{U}_0\rightarrow \fn{I}_\text{a}}$ pólusai:
\begin{equation}
	p_1 = -\frac{\sqrt{\Ja\, \left(\Ja\, {\Ra}^2 - 4\, \La\, \ke\, \km\right)} + \Ja\, \Ra}{2\, \Ja\, \La}
\end{equation}
\begin{equation}
	p_2 = \frac{\sqrt{\Ja\, \left(\Ja\, {\Ra}^2 - 4\, \La\, \ke\, \km\right)} - \Ja\, \Ra}{2\, \Ja\, \La}
\end{equation}

Az időállandók a pólusok reciprokának negáltjai:
\begin{equation}
	T_1 = -\frac{2\, \Ja\, \La}{\sqrt{\Ja\, \left(\Ja\, {\Ra}^2 - 4\, \La\, \ke\, \km\right)} - \Ja\, \Ra} = 0,0145\text{ s}
\end{equation}
\begin{equation}
	T_2 = \frac{2\, \Ja\, \La}{\sqrt{\Ja\, \left(\Ja\, {\Ra}^2 - 4\, \La\, \ke\, \km\right)} + \Ja\, \Ra} = 1,3825\cdot 10^{-4}\text{ s}
\end{equation}

A rendszer nullfrekvenciás erősítése $A_{\fn{U}_0\rightarrow \fn{I}_\text{a}} = 0$.

%}}}

%{{{ A terhelőnyomaték -- szögsebbesség átviteli függvény felírása
\subsection{A terhelőnyomaték -- szögsebesség átviteli függvény felírása}
\label{subsect:Wt-w}

Az előrecsatoló és visszacsatoló át átviteli függvényei
\begin{equation}
	\fn{W}_\text{x} = -\fn{W}_\text{me},
\end{equation} és 
\begin{equation}
	\fn{W}_\text{f} = -\km\ke\fn{W}_\text{el}.
\end{equation}

A zárt hurkú átviteli függvény
\begin{equation}
	\fn{W}_{\tau_\text{t}\rightarrow\Omega_\text{ki}} = 
	\frac{\Ra + \La\, s}{\Ja\, \La\, s^2 + \Ja\, \Ra\, s + \ke\, \km}
\end{equation}

A $\fn{W}_{\fn{U}_0\rightarrow \fn{I}_\text{a}}$ pólusai:
\begin{equation}
	p_1 = \frac{\sqrt{\Ja\, \left(\Ja\, {\Ra}^2 - 4\, \La\, \ke\, \km\right)} - \Ja\, \Ra}{2\, \Ja\, \La}
\end{equation}
\begin{equation}
	p_2 = -\frac{\sqrt{\Ja\, \left(\Ja\, {\Ra}^2 - 4\, \La\, \ke\, \km\right)} + \Ja\, \Ra}{2\, \Ja\, \La}
\end{equation}

Az időállandók a pólusok reciprokának negáltjai:
\begin{equation}
	T_1 = -\frac{2\, \Ja\, \La}{\sqrt{\Ja\, \left(\Ja\, {\Ra}^2 - 4\, \La\, \ke\, \km\right)} - \Ja\, \Ra} = 1,3825\cdot10^{-4}\text{ s}
\end{equation}
\begin{equation}
	T_2 = \frac{2\, \Ja\, \La}{\sqrt{\Ja\, \left(\Ja\, {\Ra}^2 - 4\, \La\, \ke\, \km\right)} + \Ja\, \Ra} = 0,0145\text{ s}
\end{equation}

A rendszer nullfrekvenciás erősítése
\begin{equation}
	A_{\tau_\text{t}\rightarrow\Omega_\text{ki}} =
	- \frac{\Ra}{\km\ke} = -3275,9
\end{equation}


%}}}

%{{{ A terhelőnyomaték -- áram átviteli függvény felírása
\subsection{A terhelőnyomaték -- áram átviteli függvény felírása}
\label{subsect:Wt-i}

Az előrecsatoló és visszacsatoló át átviteli függvényei
\begin{equation}
	\fn{W}_\text{x} = \fn{W}_\text{me}\ke\fn{W}_\text{el},
\end{equation} és 
\begin{equation}
	\fn{W}_\text{f} = \km.
\end{equation}

A zárt hurkú átviteli függvény
\begin{equation}
	\fn{W}_{\tau_\text{t}\rightarrow\Omega_\text{ki}} = 
	\frac{\Ra + \La\, s}{\Ja\, \La\, s^2 + \Ja\, \Ra\, s + \ke\, \km}
\end{equation}

A $\fn{W}_{\fn{U}_0\rightarrow \fn{I}_\text{a}}$ pólusai:
\begin{equation}
	p_1 = \frac{\sqrt{\Ja\, \left(\Ja\, {\Ra}^2 - 4\, \La\, \ke\, \km\right)} - \Ja\, \Ra}{2\, \Ja\, \La}
\end{equation}
\begin{equation}
	p_2 = -\frac{\sqrt{\Ja\, \left(\Ja\, {\Ra}^2 - 4\, \La\, \ke\, \km\right)} + \Ja\, \Ra}{2\, \Ja\, \La}
\end{equation}

Az időállandók a pólusok reciprokának negáltjai:
\begin{equation}
	T_1 = -\frac{2\, \Ja\, \La}{\sqrt{\Ja\, \left(\Ja\, {\Ra}^2 - 4\, \La\, \ke\, \km\right)} - \Ja\, \Ra} = 1,3825\cdot10^{-4}\text{ s}
\end{equation}
\begin{equation}
	T_2 = \frac{2\, \Ja\, \La}{\sqrt{\Ja\, \left(\Ja\, {\Ra}^2 - 4\, \La\, \ke\, \km\right)} + \Ja\, \Ra} = 0,0145\text{ s}
\end{equation}

A rendszer nullfrekvenciás erősítése
\begin{equation}
	A_{\tau_\text{t}\rightarrow\Omega_\text{ki}} = \frac{1}{\km} = 17,1821
\end{equation}


%}}}
